\label {predefined_stuf}

\subsection {Conventions}

\paragraph{}
This section aims to describe precisely the functions used 
in REAL, so we need to begin with some words about actual 
conventions used.

\paragraph{}
All pre-defined function names are in italics. Only parameters 
type are given in the function prototype (in bold font). 
Returned type is specifiate by :\\
returns $<Type\_Name>$;

\subsubsection {Basic types}
\paragraph{}
The last part of the algorithm (final set verification) is
able to work on non-set types. Basic types must be evaluated, 
too. Here is the list of types supported (derivated from AADL 
basic types) :

\begin {itemize}
\item Integer
\item String
\item Boolean
\item Real
\end {itemize}
Classical basic operations are implemented for all thoses types.

\subsubsection{Example}
\paragraph{}
function ''func'' spec :\\ 
\textit{func} (\textbf{T\_Data}) returns boolean; 

\subsection {Sets}
\label {predefined_sets}

\paragraph{}
As seen before, it exists a predefined set corresponding to each REAL type~(\ref{types}):
\begin {itemize}
\item \textit{Data\_Set} : set of \textbf{T\_Data}
\item \textit{Subprogram\_Set} : set of \textbf{T\_Subprogram}
\item \textit{Subprogram\_Call\_Set} : set of \textbf{T\_Subprogram\_Call}
\item \textit{Sequence\_Call\_Set} : set of \textbf{T\_Sequence\_Call}
\item \textit{Thread\_Set} : set of \textbf{T\_Thread}
\item \textit{Threadgroup\_Set} : set of \textbf{T\_Threadgroup}
\item \textit{Process\_Set} : set of \textbf{T\_Process}
\item \textit{Memory\_Set} : set of \textbf{T\_Memory}
\item \textit{Processor\_Set} : set of \textbf{T\_Processor}
\item \textit{Virtual\_Processor\_Set} : set of \textbf{T\_Virtual\_Processor}
\item \textit{Bus\_Set} : set of \textbf{T\_Bus}
\item \textit{Virtual\_Bus\_Set} : set of \textbf{T\_Virtual\_Bus}
\item \textit{Connection\_Set} : set of \textbf{T\_Connection}
\item \textit{Device\_Set} : set of \textbf{T\_Device}
\item \textit{End\_To\_End\_Flows\_Set} : set of \textbf{T\_End\_To\_End\_Flow} (the flows that exist in the distributed application)
\item \textit{System\_Set} : set of \textbf{T\_System}
\item \textit{Local} : set of instances which are of the same type than 
the caller node. Note that if the theorem is actually not directly called (cf. case of library theorems~\ref {theorems}), then the \textit{Local} set is actually equal to the node which is calling the theorem actually calling the final theorem. This is named context inheritence.
\end {itemize}

\subsection {Set functions}

\paragraph{}
Set functions are basic operations in set theory, wich are :
\begin {itemize}
\item Union
\item Intersection
\item Complement
\end {itemize}  

\subsubsection {Union}

\paragraph{}
\textit{Union}, is defined by :\\ 
\textit{Union} ($<Set\_Type>$, $<Set\_Type>$) returns $<Set\_Type>$;\\
Where all $<Set\_Type>$ must be the same.

\paragraph{}
It exists a variant accepting a Generic\_Set as first parameter :\\
\textit{Union} (\textbf{T\_Generic}, $<Set\_Type>$) returns \textbf{T\_Generic};\\

\paragraph{}
Note that, as exposed before, Union is actually a potentialy 
redondant union, ie. an element present once in both parameter 
sets will be present twice in resulting set. A non-redondant version
(thus complying to set theory union) is called Distinct\_Union :\\
\textit{Distinct\_Union} ($<Set\_Type>$, $<Set\_Type>$) returns $<Set\_Type>$;\\
\textit{Distinct\_Union} (\textbf{T\_Generic}, $<Set\_Type>$) returns \textbf{T\_Generic};

\subsubsection {Intersection}

\paragraph{}
\textit{Intersection}, is defined by :\\ 
\textit{Intersection} ($<Set\_Type>$, $<Set\_Type>$) returns $<Set\_Type>$;\\
Where all $<Set\_Type>$ must be the same.

\paragraph{}
It exists a variant accepting a Generic\_Set as first parameter :\\
\textit{Union} (\textbf{T\_Generic}, $<Set\_Type>$) returns $<Set\_Type>$;

\subsubsection {Complement}

\paragraph{}
\textit{Complement}, is defined by :\\ 
\textit{Complement} ($<Set\_Type>$) returns $<Set\_Type>$;\\
Where all $<Set\_Type>$ must be the same.

\subsection {Relations}
\paragraph{}
Relations refers to existing Ocarina~\cite{HZP07} functions.\\
Note that each reference to \textit{an element} can be 
replaced by {a set}, since the function will really be 
applied to each elements of the parameter-passed sets. 
While all elements are in fact variables (in the meaning 
of a designator with fluctuating value), they are synonymous 
to their related set.\\
The range variable (ie. the element defined just after the 
\textbf{foreach} keyword) is an exception to this rule, and 
should always be refered as an element of the range set.\\
Of course, anonymous sets cannot be refered with their name 
either...

\subsubsection {Is\_Subcomponent\_Of}

\textit{Is\_Subcomponent\_Of} takes as parameters two elements,
and return true if the first element is an AADL instance which 
is a subcomponent of the second element.

\subsubsection {Is\_Bound\_To}

\textit{Is\_Bound\_To} takes as parameters two elements,
and return true if the first element is an AADL instance which 
is bound to the second element via the actual\_$*$\_binding 
property class, where $*$ can be either \textit{processor},
\textit{memory} or \textit{connection}.

\subsubsection {Is\_Connected\_To}

\textit{Is\_Connected\_To} takes as parameters two elements,
and return true if the first element is an AADL instance which 
is connected to the second element.

\subsubsection {Is\_Called\_By}

\textit{Is\_Called\_To} takes as parameters two elements,
and return true if the second element has a call sequence
with a subprogram call on the first element (which must be
a subprogram).

\subsubsection {Is\_Calling}

\textit{Is\_Calling} takes as parameters two elements,
and return true if the first element has a call sequence
with a subprogram call on the second element (which must 
be a subprogram).

\subsubsection {Is\_Accessed\_By}

\textit{Is\_Accessed\_By} takes as parameters two elements,
and return true if the first element is an AADL instance which 
is accessed by the second element, either directly or not.

Note that \textit{Is\_Accessed\_By} can be used with a 
connection instance and a data instance, returning true if the
connection's destination is actually the provided data, and false 
elsewhere.

\subsubsection {Is\_Accessing\_To}

\textit{Is\_Accessing\_To} is the same as \textit{Is\_Accessed\_By}, 
but returns the instances which are accessing (ie. elements of
the second parameter) instead of the instances which are accessed.

\subsubsection {Is\_Passing\_Through}

\textit {Is\_Passing\_Through} returns the end to end flows which are passing
through the provided component instance.


\subsection {Regular functions}
\label{check_function}

\paragraph{}

\texttt{Verification functions} are functions which can be called 
within the \texttt{verification expression}. Their is two kinds of 
verification functions :
\begin {itemize}
\item \texttt{set-based functions}, which always takes a set or
element as first parameter, and can have others parameters 
(usually literal). They always returns a value or a list of values. 
eg : Cardinal, Get\_Property\_Value, Property\_Exists;
\item \texttt{value manipulation functions}, where the single 
parameter is a value or a list of values, and which returns another 
value.
\end {itemize}

\subsubsection {Property\_Exists}

\textit {Property\_Exists} takes two parameters, an element
or a set and a string literal. If the first argument is an 
element, then it returns true if the specified property is 
defined on this element, and else in the other case.
If the first argument is a set, it returns true if all the
the specified property is defined on all the elements of the 
set. The \textit{Exists} function is an alias of \textit 
{Property\_Exists}.
 
\subsubsection {Get\_Property\_Value}

\textit {Get\_Property\_Value} takes two parameters, an element
or a set and a string literal. It returns the either the value 
of the property whose name is the second parameter in the node 
designed by the first parameter, if the first parameter is an
element, or a a list of values which are the results of the 
application of \textit {Get\_Property\_Value} on each element
of the set. The \textit {Property} function is an alias of the
\textit {Get\_Property\_Value} function.

This is currently the only way to produce a list of values.

\subsubsection {Cardinal}

\textit {Cardinal} takes a set as parameter. It returns the 
set cardinal (ie. an integer).

\subsubsection {First}

\textit{First} is a function which cann be applied on a range or on a 
list of range. It returns respectively a float element (even if the 
actual value of the AADL range term was an integer) or a list of
floats, which are the lower bounds of each range of the list.

Whenever called on a empty list, it returns another empty list. 
However, when called on a nil argument, it returns an error and
the theorem is therefore considered false.

\subsubsection {Last}

\textit{Last} is a function which can be applied on a range or on a 
list of range. It returns respectively a float element (even if the 
actual value of the AADL range term was an integer) or a list of
floats, which are the Upper bounds of each range of the list.

Whenever called on a empty list, it returns another empty list. 
However, when called on a nil argument, it returns an error and
the theorem is therefore considered false.

\subsubsection {List}

\textit{List} is a function which can be applied on a set of 
literals, or on a REAL \textit{Set}. In the first case, it
build a list that contains the values given as paremeters. In 
the second case, it build a list of \textit{Elements} which are 
the set's elements.

\subsubsection {Size}

\textit{Size} is a function which can be applied on any kind of list. 
It returns the number of elements in the list (an integer value).

\subsubsection {Head}

\textit{Head} is a function which can be applied on any kind of list. 
It returns the first element of the list (ie. on a list of integers, 
it will return an interger).

Whenever called on a empty list, it returns an error and the theorem 
is therefore considered false. Thus, test should be done using a 
ternary expression and the \textit{Size} function on the list.

\subsubsection {Queue}

\textit{Queue} is a function which can be applied on any kind of list. 
It returns the list without the first element.

Whenever called on a empty list, it returns an error and the theorem 
is therefore considered false. Thus, test should be done using a 
ternary expression and the \textit{Size} function on the list.

 \subsubsection {Max}

\textit {Max} is a function which can be applied on a list of 
floats or integers as parameter. It always return a float which 
is the highest value found. Note that the only way to produce 
the list is currently to call the Get\_Property\_Value on a list.
 
\subsubsection {Min}

\textit {Min} is a function which can be applied on a list of 
floats or integers as parameter. It always return a float which 
is the lowest value found. Note that the only way to produce the 
list is currently to call the Get\_Property\_Value on a list.
 
\subsubsection {Sum}

\textit {Sum} is a function which can be applied on a list of 
floats or integers as parameter. It always return a float which 
is the sum of all the values found. Note that the only way to 
produce the list is currently to call the Get\_Property\_Value 
on a list.

\subsubsection {Product}

\textit {Product} is a function which can be applied on a list 
of floats or integers as parameter. It always return a float 
which is the product of all the values found. Note that the 
only way to produce the list is currently to call the 
Get\_Property\_Value on a list.

\subsubsection {GCD}

\textit {GCD} is a function which can be applied on a list of 
floats or integers as parameter. It always return a float 
which is the Greatest Common Divisor of all the values 
found. Note that the only way to produce the list is 
currently to call the Get\_Property\_Value on a list.

\subsubsection {LCM}

\textit {LCM} is a function which can be applied on a list of 
floats or integers as parameter. It always return a float 
which is the Lowest Common Multiple of all the values 
found. Note that the only way to produce the list is 
currently to call the Get\_Property\_Value on a list.


\subsection {Aggregation functions}

\subsubsection {MSum}

\textit {MSum} is an aggregation function which take a return expression
as parameter. It will return the sum of the evaluations for each element 
of the range set.

\subsubsection {MMax}

\textit {MMax} is an aggregation function which take a return expression
as parameter. It will return the maximum of the evaluations on all element 
of the range set.

